%%%%%%%%%%%%%%%%%%%%%%%%%%%%%%%%%%%%%%%%%
% Make sure to set your name, legi number and url to the right git branch.
\newcommand{\hmwkAuthorName}{YOUR NAME} % Your name
\newcommand{\hmwkAuthorLegi}{YOUR LEGI NUMBER} % Your name
\newcommand{\hmwkGitBranch}{YOUR GIT BRANCH} % Your name
%
%%%%%%%%%%%%%%%%%%%%%%%%%%%%%%%%%%%%%%%%%

%----------------------------------------------------------------------------------------
%	PACKAGES AND OTHER DOCUMENT CONFIGURATIONS
%	Skip this
%----------------------------------------------------------------------------------------

\documentclass{article}

\usepackage{fancyhdr} % Required for custom headers
\usepackage{lastpage} % Required to determine the last page for the footer
\usepackage{extramarks} % Required for headers and footers
\usepackage{graphicx} % Required to insert images
\usepackage{lipsum} % Used for inserting dummy 'Lorem ipsum' text into the template

% Margins
\topmargin=-0.45in
\evensidemargin=0in
\oddsidemargin=0in
\textwidth=6.5in
\textheight=9.0in
\headsep=0.25in 

\linespread{1.1} % Line spacing

% Set up the header and footer
\pagestyle{fancy}
\lhead{\hmwkAuthorName} % Top left header
\chead{\hmwkClass\ \hmwkTitle} % Top center header
\rhead{\firstxmark} % Top right header
\lfoot{\lastxmark} % Bottom left footer
\cfoot{} % Bottom center footer
\rfoot{Page\ \thepage\ of\ \pageref{LastPage}} % Bottom right footer
\renewcommand\headrulewidth{0.4pt} % Size of the header rule
\renewcommand\footrulewidth{0.4pt} % Size of the footer rule

\setlength\parindent{0pt} % Removes all indentation from paragraphs

%----------------------------------------------------------------------------------------
%	DOCUMENT STRUCTURE COMMANDS
%	Skip this
%----------------------------------------------------------------------------------------

% Header and footer for when a page split occurs within a problem environment
\newcommand{\enterProblemHeader}[1]{
\nobreak\extramarks{#1}{#1 continued on next page\ldots}\nobreak
\nobreak\extramarks{#1 (continued)}{#1 continued on next page\ldots}\nobreak
}

% Header and footer for when a page split occurs between problem environments
\newcommand{\exitProblemHeader}[1]{
\nobreak\extramarks{#1 (continued)}{#1 continued on next page\ldots}\nobreak
\nobreak\extramarks{#1}{}\nobreak
}

\setcounter{secnumdepth}{0} % Removes default section numbers
\newcounter{homeworkProblemCounter} % Creates a counter to keep track of the number of problems

\newcommand{\homeworkProblemName}{}
\newenvironment{homeworkProblem}[1][Problem \arabic{homeworkProblemCounter}]{ % Makes a new environment called homeworkProblem which takes 1 argument (custom name) but the default is "Problem #"
\stepcounter{homeworkProblemCounter} % Increase counter for number of problems
\renewcommand{\homeworkProblemName}{#1} % Assign \homeworkProblemName the name of the problem
\section{\homeworkProblemName} % Make a section in the document with the custom problem count
\enterProblemHeader{\homeworkProblemName} % Header and footer within the environment
}{
\exitProblemHeader{\homeworkProblemName} % Header and footer after the environment
}

\newcommand{\problemAnswer}[1]{ % Defines the problem answer command with the content as the only argument
\noindent\framebox[\columnwidth][c]{\begin{minipage}{0.98\columnwidth}#1\end{minipage}} % Makes the box around the problem answer and puts the content inside
}

\newcommand{\homeworkSectionName}{}
\newenvironment{homeworkSection}[1]{ % New environment for sections within homework problems, takes 1 argument - the name of the section
\renewcommand{\homeworkSectionName}{#1} % Assign \homeworkSectionName to the name of the section from the environment argument
\subsection{\homeworkSectionName} % Make a subsection with the custom name of the subsection
\enterProblemHeader{\homeworkProblemName\ [\homeworkSectionName]} % Header and footer within the environment
}{
\enterProblemHeader{\homeworkProblemName} % Header and footer after the environment
}
   
%----------------------------------------------------------------------------------------
%	NAME AND CLASS SECTION
%	Skip this
%----------------------------------------------------------------------------------------

\newcommand{\hmwkTitle}{Sampling} % Assignment title
\newcommand{\hmwkDueDate}{Monday,\ March\ 13th\ 12:00,\ 2017} % Due date
\newcommand{\hmwkClass}{SLT coding exercise\ \#2} % Course/class
\newcommand{\hmwkClassTime}{Mo 16:15} % Class/lecture time
\newcommand{\hmwkClassInstructor}{} % Teacher/lecturer

%----------------------------------------------------------------------------------------
%	TITLE PAGE
%	Skip this
%----------------------------------------------------------------------------------------

\title{
\vspace{2in}
\textmd{\small{\hmwkClass}}\\
\textmd{\textbf{\hmwkTitle}}\\
\small{https://gitlab.vis.ethz.ch/vwegmayr/slt-coding-exercises}\\
\normalsize\vspace{0.1in}\small{Due\ on\ \hmwkDueDate}
%\vspace{0.1in}\large{\textit{\hmwkClassInstructor\ \hmwkClassTime}}
\vspace{3in}
}

\author{
\hmwkAuthorName\\
\hmwkAuthorLegi
}

\date{ } % Insert date here if you want it to appear below your name

\begin{document}

\maketitle

%----------------------------------------------------------------------------------------
%	TABLE OF CONTENTS
%	Skip this
%----------------------------------------------------------------------------------------

%\setcounter{tocdepth}{1} % Uncomment this line if you don't want subsections listed in the ToC

\newpage
\tableofcontents
\newpage

%----------------------------------------------------------------------------------------
%	SECTIONS
%	Now you are in the right hood
%----------------------------------------------------------------------------------------

\begin{homeworkProblem}[The Model]
The model section is intended to allow you to recapitulate the essential ingredients used in \hmwkTitle. Write down the \textit{necessary} equations to specify \hmwkTitle\ and and shortly explain the variables that are involved. This section should only introduce the equations, their solution should be outlined in the implementation section.\newline
Try to present the model or method as if you had to explain it to somebody. This way you can make sure that you understood everything that is necessary to provide a coherent explanantion.
\newline
Hard limit: One page
\vspace{10pt}

\problemAnswer{ % Answer
Your Answer
}
\end{homeworkProblem}
\clearpage

%----------------------------------------------------------------------------------------
\begin{homeworkProblem}[The Questions]
This is the core section of your report, which contains the tasks for this exercise and your respective solutions. Make sure you present your results in an illustrative way by making use of graphics, plots, tables, etc. so that a reader can understand the results with a single glance. Check that your graphics have enough resolution or are vector graphics. Consider the use of GIFs when appropriate.\newline
Hard limit: Two pages (if this is too little, you may move stuff to the ``Your Page'' section)

\begin{homeworkSection}{(a) Sampling for Optimization}
In order to get a basic idea about sampling, we will use it for the prototypical Ising model from Physics but apply it to image denoising. Consider a noisy, binary image $h=(h_1,\dots,h_n)$ where $h_i\in\{\pm1\}$ is the value of the i-th pixel. The Ising model tries to find a denoised image $\sigma$ by minimizing the following energy function:
$$E(\sigma) = -\sum_{i=1}^{n}\sigma_i\left(h_i+J\sum_{j\in N_i}\sigma_j\right)$$
where $N_i$ is the set of neighbors of pixel $i$ (for a 2D image that is either the 4- or 8-neighborhood).
\newline\newline
First find yourself an image ($\sim$300x300) which has some interesting structure and convert it into a binary image $g$ (i.e. black and white), which will serve as the ground truth.
From $g$ create a noisy image $h$ by flipping a random subset of pixels. 
We define the loss $L$ of a reconstruction $\sigma$ by $L(\sigma)=n-\sum_{i=1}^{n}\sigma_ig_i$.

\paragraph{Metropolis \& Heatbath} Implement these two sampling algorithms discussed in the exercise and plot $L$ and $E$ as a funtion of time/iteration for both of them.
Can you think of other interesting properties that one can measure during the sampling process?

\paragraph{Simulated Annealing} Implement simulated annealing for Metropolis and Heatbath by starting at a low $\beta$ and gradually increasing it. Try different annealing schedules (e.g. linear, logarithmic, polynomial increase of $\beta$).

\paragraph{Parallel Tempering} is another interesting simulation method, see  its wikipedia page for a quick introduction. Try to implement what they refer to at the end as ``super simulated annealing'', i.e. there are two systems with increasing $\beta$, but one is lagging behind a bit and tries to inject its lower $\beta$ into the system with higher $\beta$.

\paragraph{Remarks} Consider the paricular structure of the Ising model - how can you make use of the (in-)dependence of pixels in order to e.g. make computation more efficient? A cool visualization is to use GIFs or other ``video'' formats to show the reconstructed image $\sigma$ as it evolves during the optimization.
\end{homeworkSection}
\pagebreak
\begin{homeworkSection}{(b) Sampling for Expectations}
If you found the Ising model kind of lame, then this is the right section for you. Another application of MCMC is to compute the expected value of a random variable - as for instance in Restricted Boltzmann Machines (RBMs), which allow for efficient Gibbs Sampling. Please refer to the reference for more details about them, it even provides the code ;) Only so much for now: they played a crucial role in the early development of deep learning (early as 2006 early).
\paragraph{Implement} an RBM using the MNIST data set from last time and apply Locally Linear embedding to the representations obtained from the RBM. Does your embedding get better? You may also try to stack RBMs as described in the reference and see how the embedding changes as you get higher level representations.
\end{homeworkSection}

\begin{section}{References}
http://www.cs.princeton.edu/courses/archive/spr06/cos598C/papers/AndrieuFreitasDoucetJordan2003.pdf\newline
http://www.cs.cmu.edu/$\sim$kmuruges/Home\_files/mcmc.pdf \newline
https://en.wikipedia.org/wiki/Parallel\_tempering \newline
http://deeplearning.net/tutorial/rbm.html
\end{section}

\vspace{10pt}
\problemAnswer{ % Answer
Your Answer
}
\end{homeworkProblem}
\clearpage

%----------------------------------------------------------------------------------------
\begin{homeworkProblem}[The Implementation]
In the implementation section you give a concise insight to the practical aspects of this coding exercise. It mainly mentions the optimization methods used to solve the model equations. Did you encounter numerical or efficiency problems? If yes, how did you solve them?
Provide the link to your git branch of this coding exercise.\newline
Hard limit: One page

\vspace{10pt}
\problemAnswer{ % Answer
Your Answer
}
\end{homeworkProblem}
\clearpage

%----------------------------------------------------------------------------------------
\begin{homeworkProblem}[Your Page]
Your page gives you space to include ideas, observations and results which do not fall into the categories provided by us. You can also use it as an appendix to include things which did not have space in the other sections.\newline
No page limit.

\vspace{10pt}
\problemAnswer{ % Answer
Your Answer

\hmwkGitBranch % defined in line 5
}
\end{homeworkProblem}
\clearpage

\end{document}

