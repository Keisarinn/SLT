

%----------------------------------------------------------------------------------------
%	PACKAGES AND OTHER DOCUMENT CONFIGURATIONS
%	Skip this
%----------------------------------------------------------------------------------------

\documentclass{article}

\usepackage[utf8]{inputenc}
\usepackage{fancyhdr} % Required for custom headers
\usepackage{lastpage} % Required to determine the last page for the footer
\usepackage{extramarks} % Required for headers and footers
\usepackage{graphicx} % Required to insert images
\usepackage{lipsum} % Used for inserting dummy 'Lorem ipsum' text into the template
\usepackage{amsmath}
\usepackage{array}
\usepackage{subcaption}


%%%%%%%%%%%%%%%%%%%%%%%%%%%%%%%%%%%%%%%%%
% Make sure to set your name, legi number and url to the right git branch.
\newcommand{\hmwkAuthorName}{Nicolas Känzig} % Your name
\newcommand{\hmwkAuthorLegi}{12-916-615} % Your name
\newcommand{\hmwkGitBranch}{12-916-615/1\_locally\_linear\_embedding} % Your name
%
%%%%%%%%%%%%%%%%%%%%%%%%%%%%%%%%%%%%%%%%%

\newenvironment{conditions}
  {\par\vspace{\abovedisplayskip}\noindent\begin{tabular}{>{$}l<{$} @{${}={}$} l}}
  {\end{tabular}\par\vspace{\belowdisplayskip}}
  

% Margins
\topmargin=-0.45in
\evensidemargin=0in
\oddsidemargin=0in
\textwidth=6.5in
\textheight=9.0in
\headsep=0.25in 

\linespread{1.1} % Line spacing

% Set up the header and footer
\pagestyle{fancy}
\lhead{\hmwkAuthorName} % Top left header
\chead{\hmwkClass\ \hmwkTitle} % Top center header
\rhead{\firstxmark} % Top right header
\lfoot{\lastxmark} % Bottom left footer
\cfoot{} % Bottom center footer
\rfoot{Page\ \thepage\ of\ \pageref{LastPage}} % Bottom right footer
\renewcommand\headrulewidth{0.4pt} % Size of the header rule
\renewcommand\footrulewidth{0.4pt} % Size of the footer rule

\setlength\parindent{0pt} % Removes all indentation from paragraphs

%----------------------------------------------------------------------------------------
%	DOCUMENT STRUCTURE COMMANDS
%	Skip this
%----------------------------------------------------------------------------------------

% Header and footer for when a page split occurs within a problem environment
\newcommand{\enterProblemHeader}[1]{
\nobreak\extramarks{#1}{#1 continued on next page\ldots}\nobreak
\nobreak\extramarks{#1 (continued)}{#1 continued on next page\ldots}\nobreak
}

% Header and footer for when a page split occurs between problem environments
\newcommand{\exitProblemHeader}[1]{
\nobreak\extramarks{#1 (continued)}{#1 continued on next page\ldots}\nobreak
\nobreak\extramarks{#1}{}\nobreak
}

\setcounter{secnumdepth}{0} % Removes default section numbers
\newcounter{homeworkProblemCounter} % Creates a counter to keep track of the number of problems

\newcommand{\homeworkProblemName}{}
\newenvironment{homeworkProblem}[1][Problem \arabic{homeworkProblemCounter}]{ % Makes a new environment called homeworkProblem which takes 1 argument (custom name) but the default is "Problem #"
\stepcounter{homeworkProblemCounter} % Increase counter for number of problems
\renewcommand{\homeworkProblemName}{#1} % Assign \homeworkProblemName the name of the problem
\section{\homeworkProblemName} % Make a section in the document with the custom problem count
\enterProblemHeader{\homeworkProblemName} % Header and footer within the environment
}{
\exitProblemHeader{\homeworkProblemName} % Header and footer after the environment
}

\newcommand{\problemAnswer}[1]{ % Defines the problem answer command with the content as the only argument
\noindent\framebox[\columnwidth][c]{\begin{minipage}{0.98\columnwidth}#1\end{minipage}} % Makes the box around the problem answer and puts the content inside
}

\newcommand{\homeworkSectionName}{}
\newenvironment{homeworkSection}[1]{ % New environment for sections within homework problems, takes 1 argument - the name of the section
\renewcommand{\homeworkSectionName}{#1} % Assign \homeworkSectionName to the name of the section from the environment argument
\subsection{\homeworkSectionName} % Make a subsection with the custom name of the subsection
\enterProblemHeader{\homeworkProblemName\ [\homeworkSectionName]} % Header and footer within the environment
}{
\enterProblemHeader{\homeworkProblemName} % Header and footer after the environment
}
   
%----------------------------------------------------------------------------------------
%	NAME AND CLASS SECTION
%	Skip this
%----------------------------------------------------------------------------------------

\newcommand{\hmwkTitle}{Locally Linear Embedding} % Assignment title
\newcommand{\hmwkDueDate}{Monday,\ March\ 6th,\ 2017} % Due date
\newcommand{\hmwkClass}{SLT coding exercise\ \#1} % Course/class
\newcommand{\hmwkClassTime}{Mo 16:15} % Class/lecture time
\newcommand{\hmwkClassInstructor}{} % Teacher/lecturer

%----------------------------------------------------------------------------------------
%	TITLE PAGE
%	Skip this
%----------------------------------------------------------------------------------------

\title{
\vspace{2in}
\textmd{\small{\hmwkClass}}\\
\textmd{\textbf{\hmwkTitle}}\\
\small{https://gitlab.vis.ethz.ch/vwegmayr/slt-coding-exercises}\\
\normalsize\vspace{0.1in}\small{Due\ on\ \hmwkDueDate}
%\vspace{0.1in}\large{\textit{\hmwkClassInstructor\ \hmwkClassTime}}
\vspace{3in}
}

\author{
\hmwkAuthorName\\
\hmwkAuthorLegi
}

\date{ } % Insert date here if you want it to appear below your name

\begin{document}

\maketitle

%----------------------------------------------------------------------------------------
%	TABLE OF CONTENTS
%	Skip this
%----------------------------------------------------------------------------------------

%\setcounter{tocdepth}{1} % Uncomment this line if you don't want subsections listed in the ToC

\newpage
\tableofcontents
\newpage

%----------------------------------------------------------------------------------------
%	SECTIONS
%	Now you are in the right hood
%----------------------------------------------------------------------------------------

\begin{homeworkProblem}[The Model]
The model section is intended to allow you to recapitulate the essential ingredients used in \hmwkTitle. Write down the \textit{necessary} equations to specify \hmwkTitle\ and and shortly explain the variables that are involved. This section should only introduce the equations, their solution should be outlined in the implementation section.\newline
Hard limit: One page
\vspace{10pt}

\problemAnswer{ % Answer
\begin{align}
\textrm{\textbf{Reconstruction Error}}\qquad \mathcal{E}(W) = \sum_{i}|\vec{X_{i}}- \sum_{j}W_{ij}\vec{X_{j}}|^2
\end{align}
\begin{conditions}
 \ensuremath{\vec{X_{i}}}     &  Data Points, real-valued with dimensionality D \\
 W     &  Weight Matrix. \ensuremath{W_{ij}} summarizes the contribution of the \textit{j}th data point to the \textit{i}th reconstruction \\   
\end{conditions}

\begin{align}
\textrm{\textbf{Embedding Error}}\qquad \Phi(Y) = \sum_{i}|\vec{Y_{i}}- \sum_{j}W_{ij}\vec{Y_{j}}|^2
\end{align}
\begin{conditions}
 \ensuremath{\vec{Y_{i}}}     &  Embedded representation of \ensuremath{\vec{X_{i}}}, real-valued with dimensionality d \\ 
\end{conditions}

Constraints:
\begin{align}
\sum_{j}W_{ij} = 1
\end{align}
\begin{align}
W_{ij} = 0 \textrm{ if } \vec{X_{j}} \textrm{ and } \vec{X_{i}} \textrm{ do not belong to the same set}
\end{align}

}
\end{homeworkProblem}
\clearpage

%----------------------------------------------------------------------------------------
\begin{homeworkProblem}[The Questions]
This is the core section of your report, which contains the tasks for this exercise and your respective solutions. Make sure you present your results in an illustrative way by making use of graphics, plots, tables, etc. so that a reader can understand the results with a single glance. Check that your graphics have enough resolution or are vector graphics. Consider the use of GIFs when appropriate.\newline
Hard limit: Two pages

\begin{homeworkSection}{(a) Get the data}
For this exercise we will work with the MNIST data set. In order to learn more about it and download it, go to http://yann.lecun.com/exdb/mnist/.
\end{homeworkSection}

\begin{homeworkSection}{(b) Locally linear embedding}
Implement the LLE algorithm and apply it to the MNIST data set. Provide descriptive visualizations for 2D \& 3D embedding spaces. Is it possible to see clusters?
\end{homeworkSection}

\begin{homeworkSection}{(c) Cluster structure}
Investigate the cluster structure of the data. Can you observe block structures in the $M$ matrix (use matrix plots)? Also plot the singular values of $M$. Do you notice something?
Can you think of ways to determine the optimal embedding dimension?
\end{homeworkSection}

\begin{homeworkSection}{(d) Nearest Neighbors}
Investigate the influence of the choice of how many nearest neighbors you take into account. Additionally, try different metrics to find the nearest neighbors (we are dealing with images!).
\end{homeworkSection}

\begin{homeworkSection}{(e) Linear manifold interpolation}
Assume you pick some point in the embedding space. How can you map it back to the original (high dimensional) space? Investigate how well this works for points within and outside the manifold (does it depend on the dimensionality of the embedding space?) Try things like linearly interpolating between two embedding vectors and plot the sequence of images along that line. What happens if you do that in the original space?
\end{homeworkSection}

\vspace{10pt}
\newpage
\begin{homeworkSection}{(b) Locally linear embedding}
The LLE algorithm was applied on a subset of a thousand images from the provided training set.
Fig. \ref{k10_embeddings} shows the calculated embeddings into 2-dimensional and 3-dimensional spaces. Clearly a clustured structure can be observed, however not all of them are well separable.

\begin{figure*}[h]
    \captionsetup[subfigure]{justification=centering}  
    \centering
    \begin{subfigure}[t]{0.5\textwidth}
        \centering
        \includegraphics[width=6cm]{figures/Emb_k10_2d.png}
        %\caption{Selfmade Cantenna}
    \end{subfigure}%
    ~ 
    \begin{subfigure}[t]{0.5\textwidth}
        \centering
        \includegraphics[width=6.2cm]{figures/Emb_k10_3d.png}
        %\caption{Professional Cantenna}
    \end{subfigure}
    \caption{Directionality measurements}
    \label{k10_embeddings}
\end{figure*}
\end{homeworkSection}

\begin{homeworkSection}{(c) Cluster structure}
When plotting the $M$ Matrix a strongly diagonal structure can be observed. As proven in Series 1 (Problem 1 - (4)), the embedding error is minimal when the embedded vectors are associated with the eigenvectors corresponding to the smallest d+1 eigenvalues of $M$. So for an optimal embedding dimension d, one has to ensure that the Matrix $M$ provides at least d+1 small eigenvalues. The closer these values are to zero, the better.

\end{homeworkSection}


\begin{homeworkSection}{(d) Nearest Neighbors}
The influence of the number of the nearest neighbors that are added to a set was analyzed by comparing the embeddings calculated using different values. The embedded vectors for the values k = [3, 20, 40] are depicted in Figures \ref{fig:awesome_image1}-\ref{fig:awesome_image3}. An Euclidean distance-metric has been used for finding the k nearest neighbors. It can be seen that when k is chosen too big the cluster structure gets lost.

\begin{figure}[!htb]
\minipage{0.32\textwidth}
  \includegraphics[width=\linewidth]{figures/Emb_k3_3d.png}
  \caption{k=3}\label{fig:awesome_image1}
\endminipage\hfill
\minipage{0.32\textwidth}
  \includegraphics[width=\linewidth]{figures/Emb_k20_3d.png}
  \caption{k=20}\label{fig:awesome_image2}
\endminipage\hfill
\minipage{0.32\textwidth}%
  \includegraphics[width=\linewidth]{figures/Emb_k40_3d.png}
  \caption{k=40}\label{fig:awesome_image3}
\endminipage
\end{figure}
\end{homeworkSection}

\begin{homeworkSection}{(e) Linear manifold interpolation}
To reconstruct an image from the embedding space first one has to find the k nearest neighbors of the embedded image to be reconstructed. Second the weights that minimze the reconstruction error in the embedding space must be computed. Lastly the image can be reconstructed by adding up the corresponding k neighbors in the original space using the calculated weigthts.\\
The higher the dimension of the embedded space the better the result, as less information gets lost during the dimensionality reduction.\\
Choosing a point that lies outside of the manifold leads to a less clear result, as the reconstruction error gets increased. If the point is chosen too far from a legit cluster, the reconstruction will lead to no result.
Fig. \ref{reconst} shows the reconstructions using two different values for k. When choosing k too big the result is much worse which matches our observations in part \textbf{(a)}.

\begin{figure*}[h]
    \captionsetup[subfigure]{justification=centering}  
    \centering
    \begin{subfigure}[t]{0.5\textwidth}
        \centering
        \includegraphics[width=6cm]{figures/8reconst_k10_3d.png}
        \caption{d=3 , k=10}
    \end{subfigure}%
    ~ 
    \begin{subfigure}[t]{0.5\textwidth}
        \centering
        \includegraphics[width=6.2cm]{figures/8reconst_k40_3d.png}
        \caption{d=3 , k=40}
    \end{subfigure}
    \caption{Directionality measurements}
    \label{reconst}
\end{figure*}
\end{homeworkSection}

\end{homeworkProblem}
\clearpage

%----------------------------------------------------------------------------------------
\begin{homeworkProblem}[The Implementation]
In the implementation section you give a concise insight to the practical aspects of this coding exercise. It mainly mentions the optimization methods used to solve the model equations. Did you encounter numerical or efficiency problems? If yes, how did you solve them?
Provide the link to your git branch of this coding exercise.\newline
Hard limit: One page

\vspace{10pt}
\problemAnswer{ % Answer
For implementing the LLE algorithm, methods provided by sklearn and numpy libraries have been used. The complexity of LLE increases quadratically (Partial eigenvalue decomposition) with the number of training data points. So to limit the computational effort, instead of using the whole provided training set containing 60k elements, a 1k subset was used for the embeddings.

}
\end{homeworkProblem}
\clearpage

%----------------------------------------------------------------------------------------
\begin{homeworkProblem}[Your Page]
Your page gives you space to include ideas, observations and results which do not fall into the categories provided by us. You can also use it as an appendix to include things which did not have space in the other sections.\newline
No page limit.

\vspace{10pt}
\problemAnswer{ % Answer
Your Answer

\hmwkGitBranch % defined in line 5
}
\end{homeworkProblem}
\clearpage

\end{document}

